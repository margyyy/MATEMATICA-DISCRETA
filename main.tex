\documentclass[article,12pt]{book}

% Pacchetti utili
\usepackage[utf8]{inputenc}
\usepackage{amsmath, amssymb, amsthm}  % Per formule matematiche
\usepackage{graphicx}                  % Per inserire immagini e grafici
\usepackage{hyperref}                  % Per collegamenti ipertestuali
\usepackage{fancyhdr}                  % Per intestazioni e piè di pagina
\usepackage{geometry}                  % Per impostare i margini
\usepackage{tikz}
\usepackage{soul}
\usepackage{xcolor}
\usepackage{scalerel}
\usepackage{relsize}


% Impostazioni per il layout
\geometry{margin=1in}
\pagestyle{fancy}
\fancyhf{}
\fancyhead[LE,RO]{\thepage}
\fancyhead[LO]{\rightmark}
\fancyhead[RE]{\leftmark}

% Ambienti per teoremi, definizioni, dimostrazioni
\newtheorem{theorem}{Teorema}[chapter]
\newtheorem{definition}{Definizione}[chapter]
\newtheorem{proposition}{Proposizione}[chapter]
\renewcommand{\contentsname}{Sommario}

\begin{document}

\title{\textbf{MATEMATICA DISCRETA}}  % Titolo
\author{SCRITTO DA MARGYYY}  % Autore
\date{DIPARTIMENTO DI INFORMATICA UNIVR \\ ANNO 2024/2025}  % Informazioni aggiuntive

\maketitle

\tableofcontents

\newpage
\section{Prefazione}
Studente del primo anno di \textit{Informatica}, date le mie lacune in matematica, per rimanere al passo ho deciso di scrivere questo libro sulla \textit{MATEMATICA DISCRETA} \\
Oltre a servire a me, rendo libera la condivisione di questo pdf come strumento di supporto per le future matricole. \\
\textbf{Questo pdf verrà condiviso su github,e con lui il codice sorgente. \\ Attenzione però, saranno consentite modifiche se e solo se verranno approvate dal sottoscritto. Tutto ciò perche questo elaborato è stato creato per un supporto mio allo studio, ma ciò non significa che non possa essere usato anche da altri. \\
Perciò detengo che sia meglio evitare di contaminare troppo questo documento con altre menti e modi diversi di ragionare, pensare, spiegare e comprendere. \\
Altri esempi, correzioni e semplificazioni saranno sicuramente graditi!}

\begin{center} \\[25ex]
    {\scalebox{50}{$\pi$}}  % Scala il simbolo pi greco 10 volte
\end{center}





\newpage
\section{Teoria degli insiemi}
Per definizione l'insieme è una \textit{collezione} di elementi che condividono una proprietà in comune.
Vediamo assieme le varie tipologie e (le più comuni) proprietà che possiamo trovare all'interno del mondo degli insiemi :

\begin{enumerate}
    \item \textbf{Appartiene} \\[1ex]
    Si dice in un insieme che un elemento \( x \) appartiene ad un insieme \( A \) tramite la notazione:  \( x \in A \).
    
    Se volessimo esprimere questa annotazione in un modo più formale potremmo rappresentarla semplicemente con:
    
    \begin{center}
        \( A = \{ x \,|\, x \text{ ha una certa proprietà} \} \)
    \end{center}

    \item \textbf{Non appartiene} \\[1ex]
    Si dice che un elemento \( y \) non appartiene all'insieme \( A \) quando: 
    \begin{center}
        \( y \notin A \), oppure per esprimere una semplice negazione \( \rightarrow \) "non \( y \in A \)"
    \end{center}
\end{enumerate}

\subsubsection{I connettivi} \\[1ex]
Parliamo adesso invece dei connettivi, simboli usati nell'ambito matematico per aiutarci a scrivere notazioni in modo \textit{rigoroso}

\begin{itemize}
    \item \( \land \rightarrow \) Congiunzione/And \( \rightarrow \) \( P \land Q \) 
    (È vera solo se soddisfa sia \( P \) che \( Q \))
    
    \item \( \lor \rightarrow \) Disgiunzione/Or \( \rightarrow \) \( P \lor Q \)  
    (Possono essere vere entrambe)
    
    \item \( \implies \) Implicazione \( \rightarrow \) \( P \) implica \( Q \)
    (\( P \implies Q \))

    \item $\iff$ Se e solo se \rightarrow \(P\) \iff \(Q\) (P è vera se solo se lo è anche Q)

\end{itemize}
\\[1ex]

\subsubsection{Vero o falso} \\[1ex]
In matematica si tende a verificare \textit{il vero}, ma bisognerebbe anche a riconoscere il \textit{falso}.
I simboli da usare per esprimere queste annotazioni sono :
\begin{itemize}
    \item ¬ \( \rightarrow \) per indicare ¬P (P non soddisfa la proprietà)
    \item \bot \( \rightarrow \) per indicare \( \bot \) (simbolo del falso)
\end{itemize}

\subsubsection{I quantificatori}
I quantificatori in matematica servono per identificare quanto sia vera una affermazione.
Abbiamo 2 tipi di quantificatori: 

\begin{itemize}
    \item Esistenziale = \(\exists x \, P(x) \) \(\rightarrow \) esiste un elemento "x" per cui vale P (la proprietà)

\end{itemize}
Proviamo a fare un esempio più pratico scrivendo:
\begin{center}
    Esiste una \( x \) appartenente all'insieme dei numeri reali \( \mathbb{R} \) tale che:

\[
\exists x \in \mathbb{R} \quad \text{tale che} \quad x^2 = 4 
\] 
Esiste un elemento $x$ tale che \(x = 4 \)
\end{center}
\begin{itemize}
    \item Universale = \(\forall\) \(\rightarrow\) indica che la proprietà è vera per \textit{tutti} gli elementi di un insieme:
    \begin{center}
        \(\forall x, P(x) \) \(\rightarrow\) Per ogni $x$ vale la proprietà  
    \end{center}
\end{itemize}
Proviamo sempre a fare un altro esempio:
\begin{center}
    \(\forall x \in \mathbb{R}, x^2 \geq 0 \)
\end{center}


\subsection{Contenimento e sottoinsieme}
\begin{enumerate}
    \item A è sottoinsieme \textbf{improprio} (contiene se stesso) di B \( \rightarrow \) \( A \subseteq B \) se e solo se:
    \begin{center}
        \(\forall x. (x \in A \implies x \in B) \)
    \end{center}
    Per dire in poche parole che se \(x \in A\) allora \(x \in B\)
    \item \(A = B\) viene indicato per esprimere una \textbf{egualianza}, dove se volessimo annotarla scriveremo 
    \begin{center}
        \(\forall x. (x \in A \iff x \in B) \)
    \end{center}
    Questa però non è l'unica annotazione che può esprimere un'egualianza.
    Per renderla ancora più evidente, usando il concetto di contenimento, possiamo esprimerla come:
    \begin{center}
        \(A \subseteq B\) e \(B \subseteq A\)
    \end{center}
    \item A è sottoinsieme \textbf{prorpio} di B \(\rightarrow\) \(A \subset B \) (tutti gli elementi di $A$ appartengono a $B$)
    Attenzione però, ciò non vuol dire per forza che $A$ sia uguale a $B$. Facciamo qualche esempio:
    \begin{center}
        \(A = \ \{1,2,3\}, B = \{1,2,3,4,5\}\)
    \end{center}
    Quindi possiamo confermare che $A$ sia \(\neq \) $B$

    \subsubsection{Principio di comprensione/specificazione}
    Adesso, grazie alle definizioni date in precendenza sui contenimenti e sottoinsiemi, possiamo affermare che: Per ogni proprietà che definisce un insieme, esiste un insieme che contiene \textit{tutti} e \textit{solo quei} elementi che la soddisfano.
    \begin{center}
        A = \{$x$$|$$P(x)$\} \(\rightarrow\) $A$ contiene tutte le $x$ per cui vale la prorpietà $P$
    \end{center}
    Volendo fare un esempio più pratico possiamo scrivere:
    \begin{center}
        \(A = \{x|x \in \mathbb{R} \land x > 2)\}\
    \end{center}
    \subsubsection{L'insieme vuoto}
        Per dire che in un insieme non ci sono elementi si usa la annotazione: 
        \begin{center}
            \(\forall x. x \notin \emptyset \)
        \end{center}
        \subsubsection{Prorprietà}
        \begin{enumerate}
            \item \( A \subseteq A \) (ogni insieme è contenuto in se stesso)
            \item \( X \subseteq Y \land Y \subseteq X \) \(\rightarrow\) questo può indicare anche una egualianzia.
            \item \(\forall A. \emptyset \subseteq A\) \(\rightarrow\) L'insieme vuoto è sottoinsieme di qualsiasi altro insieme
        \end{enumerate}
    \subsection{Principi di dimostrazione}
Una dimostrazione matematica è un processo deduttivo che, a partire da ipotesi considerate valide o da proposizioni già dimostrate, stabilisce la verità di una nuova affermazione basandosi esclusivamente sulla correttezza logica del ragionamento.
Per adesso introdurremo solo un modo per dimostrare, cioè quello per assurdo:
    \\[1ex]
Voglio dimostrare la verità di una assegnazione di $P$:
    \begin{enumerate}
        \item Suppongo $P$ sia falsa \textbf{per assurdo}
        \begin{center}
            Es: $P$ è sia pari che dispari
        \end{center}
        \item Se dal punto $1$ arrivo ad una contraddizione, allora $P$ è vera
    \end{enumerate}
\subsubsection{Attenzione!}
Supponiamo che \(A = \{1,2\} \)
    \begin{itemize}
        \item \(\emptyset \subseteq A = vera \)
        \item \(\emptyset \subset  A = vera \) perchè $A$ contiene elementi
    \end{itemize}    
Ma attenzione, se avessimo:
    \begin{itemize}
        \item Se $A = \{1,2\}$
    \begin{center}
        \begin{itemize}
            \item \(\emptyset \subseteq A = vera\)
            \item \(\emptyset \subset A = falsa !\)
        \end{itemize}
    \end{center}
    \end{itemize}

\subsection{Operazione fra insiemi}
Prima di parlare di operazioni fra insiemi dobbiamo prima fare un appunto.
Bisognamo prima distinguere 2 tipi casi dove è possibile fare operazioni:
    \begin{itemize}
        \item \textbf{n upla}(+ di 2 insiemi) \ \(\rightarrow\) ma che tratteremo più avanti
        \item \textbf{coppie}, che a loro volta si distinguono in:
    \begin{center}
        \begin{enumerate}
            \item Coppia non ordinata \(\rightarrow\) $A, B$ \(\rightarrow \) $\{A,B\}$ \(\rightarrow\) $\{A,B\} = \{B,A\}$
            \item Coppia ordinata \(\rightarrow\) $X,Y,Z / X_1, X_2, X_3, \ldots$
            Dove dati 2 insiemi $X,Y$ abbiamo: \\[2ex]
            \begin{center}
                \((X,Y) = \{\{X\},\{X,Y\}\}\)
            \end{center}
            
            \begin{center}
                \textit{In un gergo più elementare possiamo dire che: Prima "prendiamo" l'insieme $X$ e poi gli aggiungiamo l'insieme $Y$}
            \end{center}
        
        
        
        
        \end{enumerate}
    \end{center}
    \end{itemize}
\newpage

\subsection{Unione insiemistica}
Per esprimere l'unione di due insiemi $X$ e $Y$ in modo rigoroso scriviamo:
    \begin{center}
        \( X \cup Y = \{a | a \in X \lor a \in Y\} \) \(\rightarrow\) $XY = X \cup Y$
        
    \end{center}
\subsubsection{Il caso ennario} \label{unione ennaria} \label{generalizzare}
    Per scrivere in modo più efficente la sequente notazione: \( X_1 \cup X_2 \cup , \dots X_n =\)
    \begin{center}
        \(\bigcup_{i=1}^{n} X_i \) \(\implies\) \((\bigcup_{i=1}^{n} X  _i)\) \(\cup \ X_n\)
    \end{center}
Proviamo a farci un esempio per rendere tutto un po' più chiaro:
\begin{itemize}
    \item Dati 3 insiemi = $A,B,C$
    \item Sia \(A \cup B \cup C\), che equivale a dire \(X_1 \cup X_2 \cup X_3\)
    \item Lo possiamo quindi esprimere come : 
    \begin{center}
        \(\bigcup_{i=1}^{3} X_i \)
            \begin{itemize}
                \item Dove $3$ rappresenta il numero degli insiemi
                \item $i=1$ equivale a dire che iniziamo a contare ad incremento di $+1$
                \item $X_i$ vuol dire contare gli insiemi $X$
            \end{itemize}
    \end{center}
\end{itemize}
\subsubsection{Per negare l'appartenenza la rappresentiamo come:}
    \begin{center}
        \(X \notin A \cup B\)
    \end{center}

\newpage

\subsubsection{Proprietà dell'unione }
\begin{enumerate}
    \item \(X \cup \emptyset = \emptyset\)
    \item \(X \cup Y = Y \cup X \)
    \item \(X \cup Y \cup Z = ( X \cup Y) \cup Z\)
    \item \(X \cup X\)
    \item \(X \subseteq Z \ \land \)/e \(Y \subseteq Z = X \cup Y \subseteq Z \)
    \item \(X \subseteq Z \iff X \cup Z = Z \)
\end{enumerate}
\begin{center}
    Partendo dalla proprietà f), proviamo a costruire un esempio di \textit{dimostrazione}: \\[2ex]

    \(X \subseteq Z \iff X \cup Z = Z\) \\[2ex]

    Suddividiamo in due casi la dimostrazione \(\downarrow\)
    \begin{enumerate}
        \item \(X \subseteq Z =\) \(\rightarrow\) Per definizionne scriviamo \(\rightarrow\)  \(\{\exists a \in X \ \land\in Z\}\)
        Quindi banalmente se \(X \cup Z = Z\) allora \(\{\forall .a \in X \subseteq Z\}\)
        \item \(X \cup Z =\) \(\rightarrow\) Per definizione scriviamo \(\rightarrow\) \(\{a \in X \land \ a \in Z\}\) \\[ex]
        Dato che \(X \cup Z = Z\) \(\rightarrow\) \(\{a | a \in X \subseteq Z\}\)    
    \end{enumerate}
\end{center}

\subsubsection{Dimostrazione passo-passo}
\begin{center}
    Riprendiamo il caso f), questa volta proviamo a scomporre di più la dimostrazione: \\[2ex]

    \(X \subseteq Z \iff X \cup Z = Z\) \\[2ex]
\begin{itemize}
    \item \(X \subseteq Z\) \(\rightarrow\) assegnamo il valore A
    \item \(X \cup Z = Z\) \(\rightarrow\) assegnamo il valore B = C
\end{itemize}
\begin{enumerate}
    \item Consideriamo la nostra dimostrazione da sinistra verso destra per implicazione.
    \begin{center}
        se \(x \implies \) \(x \subseteq Z\) allora \(X \cup Z = Z\)
    \end{center}
Sia \(x \subseteq Z\) e sia \(y \in x \cup Z\) \\
Quindi \(y \in X \land y \in Z\) \(\rightarrow\) definizione di unione

\begin{itemize}
    \item Caso 1
    \begin{center}
        se \(y \in X\), poiche \(x \subseteq Z\) \(\rightarrow\) \(y \in Z\)
    \end{center}

    \item Caso 2
    \begin{center}
        se \(y \in Z\), banalmente segue:
    
    \end{center}
    \begin{center}
        se \(y \in Z \) allora \(Z \subseteq X \cup Z\)
    \end{center}
\end{itemize}
    \item Consideriamo adesso la nostra dimostrazione da destra verso sinistra.
    \begin{center}
        se \(X \cup Z = Z\) allora \(X \subseteq Z\)
    \end{center}
    Sia \(X \cup Z \subseteq Z\) e \(Z \subseteq X \cup 
 Z\) \(\rightarrow\) quindi tutti gli elementi di $X$ sono in $Z$
\end{enumerate}
\end{center}
\newpage

\subsubsection{Definizione di n uple non ordinate}
Sia \(\{X_1,X_2,X_3\}\) = \(\{X_1,X_2\}\) \(\cup\)\ \(\{X_3\}\) \(\rightarrow\) \(\{X_1\} \(\cup\) \(\{X_2\} \(\cup\) \(\{X_3\}

\subsubsection{Definzione di insiemistica di \(n \in \mathbb{N}\)}
Per definizione tratteremo la codifica dei numeri, in questo caso prorprio di \(n \in \mathbb{N}\). \\
La codifica di un numero è un modo per rappresentare un numero in una forma comprensibile o utilizzabile da un sistema, come un computer o un dispositivo di calcolo.\\ Esistono diversi modi per codificare un numero a seconda del contesto e del tipo di sistema utilizzato. \\
Provimao a fare qualche esempio:
\begin{center}
    \[\begin{aligned} & \overline{0}:=\emptyset\\  & \overline{1}:=\{\overline{0}\}=\{\emptyset\}\\  & \overline{2}:=\{\overline{0},\overline{1}\}=\{\emptyset,\{\emptyset\}\}\\  & \overline{3}:=\{\overline{0},\overline{1},\overline{2}\}=\{\emptyset,\{\emptyset\},\{\emptyset,\{\emptyset\}\}\}\\  & \overline{n}:=\{0,\overline{1},\overline{2}\ldots\overline{n}_{-1},\overline{}\}\end{aligned}\]
\end{center}

Posiamo notare quindi che data la codifica di \textit{zero} come $= \emptyset$ \\
Ad ogni codifica del numero successivo viene assegnata la codifica esatta di ogni numero precedente ma a partire proprio dalla codifica di \textit{zero.} \\[2ex]
Come possiamo vedere nel caso della codifica di 2 $\rightarrow$ \overline{2}:=\{\overline{0},\overline{1}\}=\{\emptyset,\{\emptyset\}\}\\$

\newpage

\subsection{Intersezione insiemistica}
Per definizione di Intersezione scriviamo :
\begin{center}
    \(X \cap Y:\{z|z\in X \land z\in Y\}\)
\end{center}

Qui a differenza dell'unione insiemistica che si esprime con \textit{or}, l'intersezione la andremo a trattare con \textit{and}, perchè dati due insiemi $X$ e $Y$, prendendo in considerazione un elemento $z$, esso è presente in ciascun degli insiemi messi ad intersezione.

\subsubsection{Caso n ario}
Per esprimere una intersezione in caso n \textit{n ario} come fatto per \hyperref[unione ennaria]{l'unione}.
\begin{center}
    \(\bigcap_{i=1}^{n}X_{i}=X_{i}\cap X_2\cap\ldots X_{n}=\left(\bigcap_{i=1}^{n-i}X_{i}\right)\cap X_{n}\)
\end{center}

\subsubsection{Proprietà dell'intersezione}
\begin{enumerate}
    \item \(\times\cap\emptyset=\emptyset \)
    \item \(X \cap Y = Y \cap X\)
    \item \(X \cap (Y \cap Z) = (X \cap Z) \cap Z \)
    \item \(X \cap X = X\)
    \item se $W \subseteq X$ e $W \subseteq Y$, allora $W \subseteq X \cap Y$
    \item $X \subseteq Y \iff X \cap Y = X$
    \item \textit{Proprietà distributiva} : $X \cap (Y \cup Z) = (X \cap Y) \cup (X \cap Z)$
    \end{enumerate}


\textbf{Dimostrazione della proprietà f:}

\[
X \subseteq Y \iff X \cap Y = X
\]

\textbf{(1) Dimostriamo che } X \subseteq Y \implies X \cap Y = X:

Sia \( x \in X \). Poiché \( X \subseteq Y \), abbiamo anche \( x \in Y \). Di conseguenza, \( x \in X \cap Y \). Pertanto, ogni elemento di \( X \) appartiene anche a \( X \cap Y \), quindi \( X \subseteq X \cap Y \).

D'altra parte, per definizione dell'intersezione, \( X \cap Y \subseteq X \). Pertanto, si conclude che \( X = X \cap Y \).

\textbf{(2) Dimostriamo che } X \cap \ Y = X \implies X \subseteq Y:

Sia \( x \in X \). Dal fatto che \( X = X \cap Y \), abbiamo che \( x \in X \implies x \in X \cap Y \). Per definizione di intersezione, \( x \in X \cap Y \) implica che \( x \in Y \). Di conseguenza, ogni elemento di \( X \) appartiene anche a \( Y \), quindi \( X \subseteq Y \).

\textbf{Conclusione:} Abbiamo dimostrato che \( X \subseteq Y \implies X \cap Y = X \) e \( X \cap Y = X \implies X \subseteq Y \).

\newpage

\subsection{Differenza fra insiemi}
Per definizione di differenza fra insiemi sia:
\begin{center}
    $X \setminus Y = \{z|z \in X \land  z \notin Y$\}
\end{center}

\begin{center}
    \begin{tikzpicture}
    % Cerchio X in rosso
    \fill[grey!50] (0,0) circle(1.5) node at (-1,1.5) {$X$};
    % Cerchio Y trasparente
    \fill[white, opacity=0.5] (1.5,0) circle(1.5) node at (3,0) {$Y$};

    % Cerchio Y con contorno
    \draw[thick] (1.5,0) circle(1.5);
    
    % Linee oblique per evidenziare X \ Y
    \draw[dashed] (-1.5,0) -- (-1.5,1.5) node[midway, left] {$X \setminus Y$};
    \draw[dashed] (0.5,0) -- (0.5,1.5);
\end{tikzpicture}
\end{center}






\subsubsection{Proprietà della differenza}

\begin{enumerate}
    \item \(X \setminus \emptyset = X\)
    \item \(X \setminus X = \emptyset\)
    \item \((X \setminus Y) \cap Y = \emptyset\)
    \item \((X \setminus Y) \cup X = X\)
    \item \(X \cup Y = (X \setminus Y) \cup (X \cap Y) \cup (Y \setminus X)\)
\end{enumerate}

\subsubsection{Dimostrazione della prorpietà e:}
\begin{center}
    \(X \cup Y = (X \setminus Y) \cup (X \cap Y) \cup (Y \setminus X)\)
\end{center}

\textbf{(1) Dimostriamo per inclusione }, perciò riscriviamo:
    \begin{center}
        \(X \cup Y \subseteq (X \setminus Y) \cup (X \cap Y) \cup (Y \setminus X)\)
    \end{center}
Per ipotesi assumiamo che se $z \in X \cup Y$, allora ipotiziamo i casi:
\begin{itemize}
    \item $z \in X$
    \item $z \in Y$
\end{itemize}
Dato che $z \in X$ allora secondo la definizione di unione $z$ può appartenere o meno anche a $Y$. \\
Perciò abbiamo che :
\begin{center}
    $z \in Y$ oppure $z \notin Y$ $\rightarrow$ \textit{Principio del terzo escluso}
\end{center}

\begin{enumerate}
    \item Se $z \in X$ e $z \notin Y$ $\implies$ $z \in (X \setminus Y)$
    \item Se $z \in Y$ e $z \in X$ $\implies$ $z \in (X \cap Y)$
    \item Se $z \in Y$ e $z \notin X$ $\implies$ $z \in (Y \setminus X)$
\end{enumerate}

Abbiamo quindi dimostrato, segmentando la proprietà: 
    \begin{center}
        \(X \cup Y = (X \setminus Y) \cup (X \cap Y) \cup (Y \setminus X)\)
    \end{center}

\newpage
\subsubsection{Altre proposizioni della differenza insiemistica}
    \begin{enumerate}
        \item $(X \cup Y) \setminus Z = (X \setminus Z) \cup (Y \setminus Z)$
        \item $X \setminus (Y \setminus Z) = (X \setminus Y) \cup (X \cap Z)$
        \item $(X \setminus Y) \setminus Z = (X \setminus Y) \cup Z$
        \item $X \cap (Y \setminus Z) = X \cap Y \setminus X \cap Z$
        \item $(X \setminus Y) \cap Z = (X \cap Z) \setminus (Y \cap Z)$
    \end{enumerate}

\subsubsection{Dimostrazione della proprietà d: $X \cap (Y \setminus Z) = X \cap Y \setminus X \cap Z$}
    \begin{center}
        \begin{aligned}X \cap (Y \setminus Z) \: & =\{z|z\in X,z\in(Y \setminus Z)\}=\\  & =\{z|z\in X \land (x\in Y,z\notin z)\}\\  & =\{z|(z\in X,z\in Y) \land (z\in X \land \notin z)\}\end{aligned}
     \end{center}
     
        Sia quindi che :
        \begin{itemize}
            \item \(\{z|z \in X \land z \in Y\}\) $\rightarrow$ $z \in (X \cap Y)$
            \item $z \notin X \cap Z$ $\rightarrow$ $(X \cap Y) \setminus (X \cap Z)$
        \end{itemize}

\newpage

\subsection{Differenza simmetrica}
Sia $X,Y \rightarrow X \triangle Y$ dove la differenza simmetrica si esprime come:
\begin{center}
    $X \triangle Y = (X \setminus Y) \cup (Y \setminus X)$
\end{center}
La differenza simmetrica è molto utile in vari campi, come la logica, la teoria degli insiemi e l'informatica, per confrontare due insiemi e trovare gli elementi unici. \\
Se volessimo fare un esempio più pratico potremmo dire che sia :
    \begin{itemize}
        \item $X = \{1,2,3,4\}$
        \item $Y = \{3,4,5,6\}$
    \end{itemize}
Ora calcoliamo la differenza simmetrica :
\subsubsection{1. Calcolo di \(X \setminus Y\)}
\[
X \setminus Y = \{1, 2\} \quad (\text{elementi in } X \text{ ma non in } Y)
\]

\subsubsection{2. Calcolo di \(Y \setminus X\)}
\[
Y \setminus X = \{5, 6\} \quad (\text{elementi in } Y \text{ ma non in } X)
\]

\subsubsection{3. Differenza Simmetrica}
\[
X \Delta Y = (X \setminus Y) \cup (Y \setminus X) = \{1, 2\} \cup \{5, 6\} = \{1, 2, 5, 6\}
\]
\\[4ex]
\begin{center}
    \begin{tikzpicture}
    % Cerchio X in rosso
    \fill[grey!50] (0,0) circle(1.5) node at (-1,1.5);
    % Cerchio Y in rosso
    \fill[grey!50] (1.5,0) circle(1.5) node at (3,0) {$Y$};

    % Cerchio dell'intersezione in trasparente
    \begin{scope}
        \clip (0,0) circle(1.5);
        \fill[white] (1.5,0) circle(1.5);
    \end{scope}

    % Contorni dei cerchi
    \draw[thick] (0,0) circle(1.5);
    \draw[thick] (1.5,0) circle(1.5);
    


    % Aggiunta di un'etichetta per l'intersezione
    \node at (0.75,0) {\textcolor{black}{\textbf{X} $\triangle$ \textbf{Y}}};
\end{tikzpicture}
\end{center}
\newpage
\subsubsection{Proprietà della differenza simmetrica}

\begin{enumerate}
    \item $X \triangle Y = (X \cup Y) \setminus (X \cap Y)$
    \item $X \triangle Y = Y \triangle X$
    \item $(X \triangle Y) \triangle Z = X \triangle (Y \triangle Z)$
    \item $X \cap (Y \triangle Z) = (X \cap Y) \triangle (X \cap Z)$
    \item $X \triangle \emptyset = X$
    \item $X \triangle X = \emptyset$
    \item $(X \triangle Y) \cap Z = (X \cap Z) \triangle (Y \cap Z)$
\end{enumerate}

\subsubsection{Dimostrazione della proprietà d:}
Scriviamo la dimostrazione secondo il criterio se solo se:
\begin{center}
    \begin{align*}
        & X \cap (Y \triangle Z) \iff (X \cap Y) \triangle (X \cap Z)  \\
        &= X \cap ((Y \setminus Z) \cup (Z \setminus Y) \iff \\
        &= (X \cap (Y \setminus Z) \cup (X \cap (Z \setminus Y) \\
        &= ((X \cap Y) \ (X \cap Z)) \cup ((X \cap Z) \setminus (X \cap Y) \iff \\
        &= (X \cap Y) \triangle(Y \cap Z)
    \end{align*}
\end{center}

\newpage
\subsection{Il prodotto cartesiano}
\label{prodotto cartesiano}
Sia $X,Y$ $\rightarrow$ $X \times Y$ dove il prodotto cartesiano si esprime come :
    \begin{center}
        $X \times Y = \{(x,y)| x \in X \land y \in Y \}$
    \end{center}
Con ciò possiamo dedurre che dato un insieme $X$ elevato alla seconda avremo il suo prodotto cartesiano :
\begin{center}
    $X^2 = X \times X$ $\rightarrow$ $X^n = X \times \dots \times X$
\end{center}

Ma attenzione, dire $X \times Y \neq Y \timesX$

\begin{center}
    \begin{tikzpicture}

    % Disegna l'insieme X
    \draw[thick] (0,0) circle(1.5) node at (0, 1.8) {X};
    \node at (0, 0.5) {1};
    \node at (0, -0.5) {3};

    % Disegna l'insieme Y
    \draw[thick] (4,0) circle(1.5) node at (4, 1.8) {Y};
    \node at (4, 1) {2};
    \node at (4, 0) {4};
    \node at (4, -1) {6};

    % Frecce rosse da 1 a Y
    \draw[->, red] (0, 0.5) -- (3.5, 1);
    \draw[->, red] (0, 0.5) -- (3.5, 0);
    \draw[->, red] (0, 0.5) -- (3.5, -1);

    % Frecce blu da 3 a Y
    \draw[->, blue] (0, -0.5) -- (3.5, 1);
    \draw[->, blue] (0, -0.5) -- (3.5, 0);
    \draw[->, blue] (0, -0.5) -- (3.5, -1);

    % Disegna il prodotto cartesiano X x Y
    \draw[thick] (2, -3) circle(1.5) node at (2, -5) {$X \times Y$};
    \node at (1.3, -3.5) {(1,6)}
    \node at (1.3, -2.5) {(1,2)}
    \node at (1, -3) {(1,4)};

    %specchiato
    \node at (2.6, -3.5) {(3,6)}
    \node at (2.6, -2.5) {3,2)}
    \node at (3, -3) {(3,4)};
    
\end{tikzpicture}
\end{center}
\\
Un esempio di prodotto cartesiano può essere: 
    \begin{center}
        $\{\pi,e\}\times\{11,49\}=\{(\pi,11),(\pi,49),(e,11),(e,49)\}$
    
    \end{center}

\subsubsection{Il diagramma cartesiano}
Il prodotto cartesiano di $X$ e $Y$ determina il diagramma cartesiano stesso.
Come abbiamo detto per definizione che $x \in X$ e $y \in Y$, possimo dire quindi che ogni elemento $x$ possiamo collocarlo sull'asse orrizontale e ogni elmento $y$ sull'asse verticale.
Per correttezza in realta, per esprimere ciò che abbiamo detto in modo più rigoroso, potremmo dire che preso solo e soltanto in consideraione $\rightarrow$ $\mathbb{R}$, abbiamo: 
    \begin{center}
        $\mathbb{R}\times\mathbb{R} = \{a,b|a \in \mathbb{R}, b \in \mathbb{R}\}$
    \end{center}
\newpage
Otteniamo un piano bidimensionale infinito. Questo piano è lo spazio in cui si possono rappresentare graficamente:
    \begin{itemize}
        \item Funzioni di due variabili
        \item Curve e figure geometriche
        \item Punti, rette, parabole, circonferenze, ecc.
    \end{itemize}

\begin{center}
    
\begin{tikzpicture}
    % Disegna l'asse orizzontale (asse x)
    \draw[->] (-1, 0) -- (3, 0) node[right] {$\mathbb{R}$};
    
    % Disegna l'asse verticale (asse y)
    \draw[->] (0, -1) -- (0, 3) node[above] {$\mathbb{R}$};
    
    % Etichette per l'origine
    \node at (0, 0) [below right] {0};
\end{tikzpicture}

\end{center}

Moltiplicando invece $\mathbb{R}$ per $n, n_1$ otteniamo che:
\begin{center}
    $\mathbb{R}(n,n_1)=\mathbb{R}\times\{x|n < x < n_1\}$
\end{center}

In questo caso stiamo creando coppie di ordinate $a,b$ dove $a$ è un qualsiasi numero reale, $b$ invece appartiene all'insieme: $$\{x|n < x < n_1\}$ \\
Se volessimo rappresentarlo sul piano cartesiano avremmo che

\begin{center}
    
\begin{tikzpicture}
    % Disegna gli assi cartesiani
    \draw[->] (-1, 0) -- (4, 0) node[right] {$\mathbb{R}$};
    \draw[->] (0, -1) -- (0, 5) node[above] {$\mathbb{R}$};
    
    % Disegna le linee y = 2 e y = 3
    \draw[dashed] (-4, 2) -- (4, 2) node[right] {$y = n$};
    \draw[dashed] (-4, 3) -- (4, 3) node[right] {$y = n_1$};
    
    % Colora la striscia tra y = 2 e y = 3
    \fill[grey!20] (-4, 2) rectangle (4, 3);

    % Etichette
    \node at (4, 4.5) {Striscia tra $y = n$ e $y = n_1$};
    
\end{tikzpicture}
\end{center}

Abbiamo quindi che l'asse $X$ rappresenta tutti i numeri reali $a$, mentre invece l'asse $Y$ è \textit{limitata} tra $n,n_1$. \\
    \begin{center}
        Possiamo quindi dire := $\{(a,x\}|a\in\mathbb{R},x\in(n,n_1)\}$
    \end{center}
Concludendo abbiamo $R(n,n_1)$ che rappresenta l'insieme di punti $(a, x) \in \mathbb{R} \times \mathbb{R}$ tali che $n < x < n_1$ e $a \in \mathbb{R}$.

\subsubsection{Le coppie ordinate}
Tramite il prodotto cartesiano, siamo in grado di determinare quelle che chiamiamo coppie ordinate.
    \begin{center}
       sia $X\times Y=\{(X,Y)| a \in X \land b\in Y\}$
    \end{center}
Determinando la coppia ordinata:
\begin{center}
    $\{\{X\},\{X,Y\}\} $ dove $X$ è il primo della coppia
\end{center}
\subsubsection{Attenzione!}
Se dovessi avere una coppia $(Y,X)$, la loro coppia ordinata sarebbe:
\begin{center}
    $\{\{Y\},\{Y,X\}\} $ dove $Y$ è il primo della coppia!
\end{center}
\newpage

\subsubsection{Proprietà del prodotto cartesiano}
    \begin{enumerate}
        \item $X \times Y = \emptyset \iff X \lor Y = \emptyset$
        \item $X \times Y = Y \times X \iff X = Y$
        \item $X \times (Y \cup Z) = (X \times Y) \cup (X \times Z)$
        \item $X \times (Y \cap Z) = (X \times Y) \cap (X \times Z)$
        \item $X \times (Y \setminus Z) = \{ (x, y) \mid x \in X, \, y \in Y, \, y \notin Z \}$
        \item $X \times (Y \triangle Z) = \{ (x, y) \mid x \in X, \, y \in (Y \setminus Z) \cup (Z \setminus Y) \}$
    \end{enumerate}

\subsubsection{Dimostrazione della proprietà d:}
Sia per ipotesi $X \times (Y \cap Z)$, apriamola scrivendo:
\begin{center}
        \begin{align*}
             X \times (Y \cap Z) &= \{(x,w)|x \in X \land w \in (Y \cap Z)\} \\
                                 &= \{(x,w)|x \in X \land (w \in Y) \land (w \in Z)\} \\
                                 &= \{(x,w)|(x \in X \land w \in Y ) \land (x \in X \land w \in Z) \\
                                 &= \{(x,w)|(x \in X \land w \in Y )\} \cap \{(x,w)|(x \in X \land w \in Z )\ \\
                                 &== (X \times Y) \cap (X \times Z)
        \end{align*}
       




        
\end{center}
\newpage

\subsection{Insieme delle parti}
Per definire l'insieme delle parti (o potenza di un insieme) prendiamo in considerazione un insieme $X$, dove l'insieme delle parti di $X$ sarà denotato come $\rightarrow$ $P(X)$ o $2^S$. \\
L'insieme delle parti quindi l'insieme di tutti i sottoinsiemi di X, inclusi l'insieme vuoto e l'insieme stesso.

\subsubsection{Definzione formale}
Se $X$ è un insieme, l'insieme delle parti di $X$ sarà:
    \begin{center}
        $P(X)=\{A|A \subseteq X\}$ $\rightarrow$ dove $A$ è qualsiasi sottoinsieme di $X$
    \end{center}
\textbf{N.B}, l'insieme vuoto è contenuto in qualsiasi insieme $\rightarrow$ $\emptyset \in P(X)$, e pertanto $X \in P(X)$, quindi $X$ è contenuto in se stesso.
    \begin{center}
        \begin{align*}
        &A \subseteq X \ \ \ \ \ \ \ \ A \in P(X) \\
        &x \in X \ \ \ \ \ \ \ \ \{x\} \in P(X)
        \end{align*}
    \end{center}
\subsubsection{Esempio 1}
$X = \{a,b,c\}$ \\
$P(x)=\{\emptyset,\{a\},\{b\},\{c\},\{a,b\},\{a,c\},\{b,c\},\{a,b,c\}\}$

\subsubsection{Esempio 2}
$\overline{2}=\{0,\overline{1}\}=\{\emptyset,\{\emptyset\}\}$ \\
$P(\overline{2})=P\{\overline{0},\overline{1}\}=\{\emptyset,\{0\},\{1\},\{0,1\}\}$

\newpage
\subsection{Complemento di un sottoinsieme}
Se $U$ è un insieme e $A \subseteq U$, il complemento di A sarà definito come: \\
    \begin{center}
        $A^{c}:=U\smallsetminus A=\{x\in U\mid x\notin A\}$ e $x \in A^c \iff x \notin A$
    \end{center}
Se volessimo dare una definizone rigorosa diremmo che: Il complemento di un insieme $A$ è definito come l'insieme di tutti gli elementi che non appartengono a $A$ all'interno di un insieme universale $U$.

\subsubsection{Caratteristiche}
\begin{itemize}
    \item  È fondamentale specificare l'insieme universale $U$ per definire il complemento, poiché gli elementi considerati "fuori" da $A$ sono quelli che non appartengono a $A$ ma che sono contenuti in $U$.
    \item Il complemento dell'insieme universale $U$ è l'insieme vuoto $\emptyset$, dato che non ci sono elementi che non appartengono a $U$.
    \item  Il complemento dell'insieme vuoto $\emptyset$ è l'insieme universale $U$, poiché tutti gli elementi in $U$ non appartengono a $\emptyset$.
\end{itemize}

Se dovessimo ragiornare all'interno dei numeri, potremmo invece dire che:
\begin{center}
    $\mathbb{N} \subseteq \mathbb{Z}$
\end{center}
\begin{center}
    $\mathbb{N}^c = \mathbb{Z} \setminus \mathbb{N} =\{x \in \mathbb{Z}|x < 0\}$
\end{center}

\newpage


\subsubsection{Proprietà dei complementi}
Sia $X, A \subseteq X, B \subseteq X, A^c, B^c$ \\
    \begin{enumerate}
        \item $A \cup A^c = X$
        \item $A \cap A^c = \emptyset$
        \item $(A^c)^c = A$
        \item $X^c = \emptyset$
        \item $O^c = X$
        \item $A \setminus B = A \cap B^c$
        \item $A \subseteq B \iff B^c \subseteq A^c$
    \end{enumerate}

\subsubsection{Dimostrazione della proprietà c:}
Sia $(A^c)^c = A$, per ipotesi scriviamo: \label{dimostrazione con de morgan}
    \begin{center}
            \begin{align*}
                (A^c)^c &= \{x|x \notin A^c\}\\
                        &= \{x \in X | \neg (x \subset A^c\} \\
                        &= \{x \in X | \neg\neg (x \in A\} \\
                        &= \{x \in X | (x \in A \} = A
            \end{align*}
        
    \end{center}
Per dimostrare questa proprietà, se lo avete notato, abbiamo usato un metodo particolare, lo abbiamo fatto attraverso la \textit{Legge di De Morgan} che spiegheremo successivamente, intanto proviamo a spiegare meglio cosa è successo:

\paragraph{Definizioni:}
    \begin{itemize}
        \item $A^c$ rappresenta il complementare di $A$, cioè l'insieme di tutti gli elementi che \textbf{non} appartengono ad $A$.
        \item $(A^c)^c$ rappresenta il complementare del complementare di $A$, cioè l'insieme degli elementi che \textbf{non} appartengono a $A^c$.
 

    \end{itemize}

\paragraph{Dimostrazione:}

\textbf{1. Per mostrare che $(A^c)^c \subseteq A$:}
\begin{itemize}
    \item Sia $x \in (A^c)^c$. Per la definizione di complementare, significa che $x \notin A^c$, cioè $x \in A$.
    \item Quindi $(A^c)^c \subseteq A$.
    
\end{itemize}
\newpage 
\textbf{2. Per mostrare che $A \subseteq (A^c)^c$:}
\begin{itemize}
    \item Sia $x \in A$. Per la definizione di complementare, $x \notin A^c$, quindi $x \in (A^c)^c$.
    \item Quindi $A \subseteq (A^c)^c$.
\end{itemize}

Dalle due inclusioni $(A^c)^c \subseteq A$ e $A \subseteq (A^c)^c$, possiamo concludere che:
\[
(A^c)^c = A
\]
Questa è una proprietà fondamentale degli insiemi.

\newpage
\subsection{Legge di De Morgan}
La legge di De Morgan riguarda il modo in cui possiamo trasformare le operazioni logiche o insiemistiche quando si usano complementi, unioni e intersezioni.

In termini di insiemi, ci sono due leggi fondamentali:

\begin{itemize}
    \item \textbf{Il complementare dell'unione di due insiemi} è uguale all'\textbf{intersezione dei complementari}:
    \[
    (A \cup B)^c = A^c \cap B^c
    \]
    Questo significa che se prendi il complementare dell'unione di due insiemi, ottieni gli elementi che non appartengono né ad $A$ né a $B$, cioè l'intersezione dei loro complementari.

    \item \textbf{Il complementare dell'intersezione di due insiemi} è uguale all'\textbf{unione dei complementari}:
    \[
    (A \cap B)^c = A^c \cup B^c
    \]
    Questo significa che se prendi il complementare dell'intersezione di due insiemi, ottieni gli elementi che non appartengono a $A$ o non appartengono a $B$, cioè l'unione dei loro complementari.
\end{itemize}

\subsection*{Un esempio pratico}
Immagina due insiemi $A$ e $B$: $A$ rappresenta le persone che amano la pizza, e $B$ quelle che amano i gelati.

\begin{itemize}
    \item Secondo la prima legge di De Morgan, il complementare di "persone che amano la pizza \textbf{o} il gelato" è uguale a "persone che non amano né la pizza né il gelato".
    
    \item La seconda legge dice che il complementare di "persone che amano la pizza \textbf{e} il gelato" è uguale a "persone che non amano la pizza \textbf{o} non amano il gelato (almeno uno dei due)".
\end{itemize}

Queste leggi sono utili perché ci permettono di trasformare espressioni complesse in modi più semplici o più convenienti. \\

Rifacendoci all'esempio della \hyperref[dimostrazione con de morgan]{dimostrazione} di prima, possiamo ricavare due proprietà particolari:
\begin{center}
    
        \begin{itemize}
            \item sia $\neg(A \lor B) \equiv (\neg A) \land (\neg B)$
            \item sia $\neg(A \land B) \equiv (\neg A) \lor (\neg B)$
        \end{itemize}
        
\end{center}
\newpage
\subsection{Relazioni fra insiemi}
Per definire che cos'è che mette in relazione due insiemi, sia $X,Y$ dove la \textit{relazione} è il loro \hyperref[prodotto cartesiano]{prodotto cartesiano.}
Per definizione quindi:
    \begin{center}
        $R \subseteq X \times Y$
    \end{center}
Dove presi due elementi: \[
\underset{X}{\underset{\rotatebox[origin=c]{270}{$\in$}}{a}} \, R \, \underset{Y}{\underset{\rotatebox[origin=c]{270}{$\in$}}{b}} \iff (a,b) \in R
\] \\
R sarà incluso in $X \times X$ e $\emptyset$ incluso in $X \times Y$. \\
Con \textbf{arietà} di una relazione invece intendiamo il numero di elementi o insiemi coinvolti nella relazione stessa 
\
\subsubsection{Relazione Unaria}
Una \textbf{relazione unaria} è una relazione che coinvolge solo un singolo insieme di elementi. In altre parole, una relazione unaria può essere vista come un modo per attribuire proprietà a ogni elemento di un insieme.\\
Se \( A \) è un insieme, una relazione unaria \( R \) su \( A \) è un sottoinsieme di \( A \). Gli elementi di \( R \) sono quegli elementi di \( A \) che soddisfano una certa proprietà.

\subsubsection{Esempi}
\begin{enumerate}
    \item \textbf{Proprietà di Pari}: Considera l'insieme \( A = \{1, 2, 3, 4, 5, 6\} \). La relazione unaria ``essere un numero pari'' può essere rappresentata come \( R = \{2, 4, 6\} \). Qui, \( R \) contiene gli elementi di \( A \) che soddisfano la proprietà di essere pari.
    
    \item \textbf{Proprietà di Essere Maggiore di 3}: Se prendiamo lo stesso insieme \( A = \{1, 2, 3, 4, 5, 6\} \), la relazione unaria ``essere maggiore di 3'' sarà \( R = \{4, 5, 6\} \).
\end{enumerate}

\subsubsection{Rappresentazione}
In notazione, una relazione unaria può essere scritta come:
\[
R(x) \quad \text{dove } x \in A
\]
significa che \( x \) ha una certa proprietà descritta dalla relazione \( R \).
\\
In sintesi, una relazione unaria è un modo per descrivere e categorizzare gli elementi di un insieme in base a una singola proprietà, ed è una delle forme più semplici di relazioni in matematica.

\newpage
\subsubsection{Facciamo qualche esempio utilizzando le diagonali}
    \textit{Esempio 1} \\


\begin{center}
\begin{minipage}{0.45\textwidth}
    \begin{tikzpicture}[scale=1.5]
        % Disegno degli assi
        \draw[->] (-1, 0) -- (3, 0) node[right] {\(x\)};
        \draw[->] (0, -1) -- (0, 3) node[above] {\(y\)};
        
        % Disegno della diagonale
        \draw[blue, thick] (-1, -1) -- (3, 3) node[right] {\(y = x\)};
    \end{tikzpicture}
\end{minipage}%
\hfill
\begin{minipage}{0.45\textwidth}
    $Diag$\((\mathbb{R}) = \{(x,x) \mid x \in \mathbb{R} \} \)
    
\end{minipage} \\
sia quindi $\{(X,Y)|X=Y \}$
\end{center}
\textit{Esempio 2}

\begin{center}
\begin{minipage}{0.45\textwidth}
    \begin{tikzpicture}[scale=1.5]
        % Disegno degli assi
        \draw[->] (-1, 0) -- (3, 0) node[right] {\(x\)};
        \draw[->] (0, -1) -- (0, 3) node[above] {\(y\)};
        
        % Disegno della diagonale tratteggiata
        \draw[blue, thick, dashed] (-1, -1) -- (3, 3) node[right] {\(y = x\)};
    \end{tikzpicture}
\end{minipage}%
\hfill
\begin{minipage}{0.45\textwidth}
    $Codiag(X) = Diag (X^c)$
\end{minipage} \\
sia quindi $\{(X,Y) \in X \times X | X \neq Y \}$
\end{center}
\newpage
\subsubsection{Definizione di relazione inversa}
\begin{center}
    Sia $R^-1 = \{(y,x) \in Y \times X | (x,y) \in \mathbb{R} \}$ dove $(y,x) \in R^-1 \iff (x,y) \in R$
\end{center}

\subsubsection{Definizione di composizione}
Il composto di due relazioni (o composizione di relazioni) è un'operazione che permette di combinare due relazioni tra insiemi. \\

\begin{center}
\begin{tikzpicture}
    % Primo cerchio X
    \draw (0,0) circle(1cm);
    \node at (0,0) {\(x\)};
    \node at (0,-1.5) {\(X\)};

    % Secondo cerchio Y
    \draw (3,0) circle(1cm);
    \node at (3,0) {\(y\)};
    \node at (3,-1.5) {\(Y\)};

    % Terzo cerchio Z
    \draw (6,0) circle(1cm);
    \node at (6,0) {\(z\)};
    \node at (6,-1.5) {\(Z\)};

    % Collegamento da x a y (semicirconferenza verso l'alto)
    \draw[bend right=45] (0,0) to (3,0);
    \node at (1.5,-1.3) {\(R_1\)};

    % Collegamento da y a z (semicirconferenza verso l'alto)
    \draw[bend right=45] (3,0) to (6,0);
    \node at (4.5,-1.3) {\(R_2\)};

    % Grande collegamento blu da x a z
    \draw[blue, thick, bend left=60] (0,0) to (6,0);

    % Etichetta R_1 \circ R_2 al centro del collegamento blu
    \node at (3,2) {\(R_1 \circ R_2\)};
    
\end{tikzpicture}
\end{center}
Per definizione quindi sia:
    \begin{center}
        $R_1 \subseteq X \times Y \land R_2 \subseteq Y \times Z$
    \end{center}

Puoi pensare alla composizione come a un collegamento "concatenato" tra due elementi tramite un terzo. In altre parole, se \( a \) è in relazione con \( b \) tramite \( R_1 \) e \( b \) è in relazione con \( c \) tramite \( R_2 \), allora possiamo dire che \( a \) è in relazione con \( c \) tramite la composizione delle due relazioni. \\

Se volessimo quindi identificare il composo di due relazioni scriveremo:\\
    \begin{center}
        $R_1 \circ R_2 = \{ (x, z) \in X \times Z \mid \exists y \in Y \text{ tale che } (x, y) \in R_1 \text{ e } (y, z) \in R_2 \}$

    \end{center}
\subsubsection{Definizione di relazione chiusa rispetto $\cup,\cap$}
\begin{itemize}
    \item Sia $Diag(X) \cup Codiag(X) = X \times X \rightarrow X^2$
    \item Sia $Diag(X) \cap Codiag(X) = \emptyset$
\end{itemize}
\newpage

\subsubsection{Proprietà delle relazioni}
Sia $R \subseteq X \times Y$ \ \ \ \ \ \ \ \ \ \ \ \ $S \subseteq Y \times Z, T \subseteq Z \times W$

\begin{enumerate}
    \item $Diag(Y) \circ R = R$
    \item $R \circ Diag(X) = R$
    \item $T \circ (S \circ R) = (T \circ S) \circ R$
\end{enumerate}
\newpage

\subsection{Famiglie di insiemi}
    \begin{center}
        Sia $\overline{n}=\{0,1,2,3,\ldots,n^{-1}\}$
    \end{center}
Prendendo un insieme di indici $i \in I \neq \emptyset$ ciasciun indice  $i$ viene associato $X_i$ \\
Determinando così la famiglia di insiemi $\rightarrow$ $X = {X_i | i \in I$

\subsubsection{Esempio 1}
$I =\overline{2}=\{0,1\}$ \\
$0 \mapsto \mathbb{Z}$ \\
$1 \mapsto \mathbb{Q} \setminus \mathbb{Z}$ \\
Sia quindi un insieme: 
    \begin{center}
    $X=\{X_0,X_1\}=\{\mathbb{Z},\mathbb{Q}\setminus \mathbb{Z}\}$
    \end{center}

\subsubsection{Esempio 2}
$I = \mathbb{N}$ \\
$n \mapsto A_n = \{m \in \mathbb{N} | m \geq n$ \\
Sia quindi un insieme: 
\begin{center}
    $X = \[A_n | n \in \mathbb{N}$
\end{center}

\subsubsection{Esempio 3}
$I = X$ \\
$a \mapsto \{a \}$ \\
Sia quindi un insieme:
    \begin{center}
        $A = \{\{a\} | a \in X \}$
    \end{center}
\newpage
\subsection{Generalizzare le operazioni}
Come gia visto in \hyperref[generalizzare]{precedenza}, possiamo generalizzare le operazioni di unione e intersezione fra insiemi:
    \begin{itemize}
        \item $\bigcup x:=\bigcup_{i\in I}X_i = \{x | \exists i \in I, x \in X_i \}$
        \item $\bigcap x:=\bigcap_{i\in I}X_i = \{x ! \forall i \in I, x \in X_i \}$
        
    \end{itemize}
\subsubsection{Esercizio 1}
$\begin{aligned}&X=\{x_{0},x_1 \}=\{\mathbb{Z}, \mathbb{Q} \setminus \mathbb{Z} \} \\&\bigcup_{i\in \bar{2}}X_{i}=x_{0}\cup X_{i}=\mathbb{Q} \\&\bigcap_{i\in \bar{2}}X_{i}=X_{0}\cap X_{1}=\emptyset\end{aligned}$

\subsubsection{Esercizio 2}
$I = \mathbb{N}$
$\begin{aligned}n\longmapsto A_{n}: & =\{m\in\mathbb{N}\mid m\geq n\}\\ A_0: & =\{m\mid m\geq0\}\\ A_1: & =\{m\mid m\geq1\}\end{aligned}$ \\

Sia che $\mathbb{N} = A_o$, possiamo dire quindi: \\

$\begin{aligned}\bigcup_{m\in\mathbb{N}}A_m&=\mathbb{N}\\\bigcap_{m\in\mathbb{N}}A_m&=\emptyset\end{aligned}$

\newpage
3,14 \\[25ex]
\begin{center} 
    {\scalebox{50}{$\pi$}}  % Scala il simbolo pi greco 10 volte
\end{center}

\newpage 

\section{Le funzioni}
Sia una relazione $R \subseteq X \times Y$, in questo caso $R$ è una funzione se e solo se : 
    \begin{center}
        $\forall x \in X, \exists !y \in Y$
    \end{center}
Se $x = x_1 \implies f = (x) = f(x_1)

\subsubsection{terminologia}
    \begin{itemize}
        \item $X$ è il \textit{dominio} della funzione (dove la funzione stessa è definita)
        \item $Y$ è il \textit{codominio} della funzione
        \item $f(x)$ è \textit{l'immagine} di $x$ tramite $f$
    \end{itemize}

\begin{center}
    \begin{tikzpicture}
    % Disegno del primo cerchio X
    \draw[thick] (0,0) circle (1.5); 
    \node at (0, -0.5) {x};
    \node at (0, 0.5) {$x_1$};
    \node at (0, 1.8) {X}; % Nome del cerchio X sopra il cerchio
    
    % Disegno del secondo cerchio Y
    \draw[thick] (4,0) circle (1.5);
    \node at (4, 0) {y};
    \node at (4, 1.8) {Y}; % Nome del cerchio Y sopra il cerchio

    % Freccia da x a y
    \draw[->, thick] (0, -0.5) -- (3.9, 0);
    
    % Freccia da x_1 a y
    \draw[->, thick] (0, 0.5) -- (3.9, 0);
\end{tikzpicture}
\end{center}

Una \textbf{funzione} quindi è un modo per collegare due insiemi di numeri o oggetti, in cui a ogni elemento di un insieme (chiamato insieme di partenza o dominio) corrisponde esattamente un elemento di un altro insieme (chiamato insieme di arrivo o codominio).

In parole più semplici:
\begin{itemize}
    \item Immagina di avere un sacchetto di frutta (il dominio) e di voler creare una regola per scegliere un tipo di frutta per ogni colore (il codominio). Ad esempio, se scegli il colore "giallo", la regola può dirti di prendere una banana.
    \item Quindi, per ogni colore che scegli, c'è una sola frutta che puoi ottenere. Se provi a scegliere un colore che non è nella tua regola, non otterrai nulla.
\end{itemize}

In sintesi, una funzione è come una macchina che prende un input (un valore dall'insieme di partenza) e produce un output (un valore dall'insieme di arrivo) seguendo una regola specifica.

\newpage
\subsubsection{Esempio 1}
Sia $y_0 \in Y$, scriveremo quindi $f__y0$ $ $: x \mapsto$ tale che $\forall x f__y0$$(x) = y__0$ \\
Se lo dovessimo invece scrivere come un insime di \textit{coppie} scriveremo:
    \begin{center}
        $f__y0$ $=\{(x,y)_0 | x \in X \}$
    \end{center}
\subsubsection{Esempio 2}
Sia $id_x = \{(x,x) | x \in X \}$ \\
Se invece volessimo la $Diag(\bar{2}) = Diag (\{0,1 \}) = \{(0,0),(1,1) \}$

\subsubsection{Esempio funzione  di dirischlet}
Sia $\mathbb{R} \mapsto \mathbb{R}$, e secondo la funzone di Dirischlet:
    \begin{center}
        Dir(x)= \begin{cases}
            1 \ \text{se} \ x \in \mathbb{Q} \\[8ex]
            0 \ \text{se} \ x \notin \mathbb{Q}
        \end{cases}
    \end{center} \\

\begin{center}
    $Dir=\{(x,1)\mid x\in\mathbb{Q}\}\cup\{(x,0)\mid x\notin\mathbb{Q}\}$
\end{center}

\newpage
\subsection{Immagine e controimmagine di una funzione}
\begin{itemize}
    \item L'\textbf{immagine} di una funzione viene espressa come l'insieme dei valori che la funzione può assumere quando applicata a tutti i valori del suo dominio.
Se dovessimo dare una definizione rigorosa di immagine potremmo dire :
    \begin{center}
        sia $f: x \rightarrow y, A \subseteq X$
    \end{center}
Diremo che :
    \begin{center}
    \begin{align}
        f(A) &= \{f(x) | x \in A \} \\
             &= \{y \in Y | \exists x \in A. f(x) = y \}
    \end{align}
    \end{center}
In questo caso consideremo A come sottoinsieme del codominio,e la consideremo appunto come l'immagine di $A$ secondo $f$ 
    \begin{center}
        $f(A) \subseteq Y$
    \end{center}

    \item La \textbf{controimmagine} ( immagine inversa) di un valore rispetto a una funzione è l'insieme di tutti gli elementi del dominio che vengono mappati in quel valore specifico dal funzione. \\
    E ciò è utile per comprendere come i valori del codominio sono correlati agli input del dominio.
    Facciamo sempre un altro esempio : \\
    Sia $B \subseteq Y$
    \begin{center}
        $f^-1 (B) = \{x \in X | f(x) \in B \}$
    \end{center}
In sostanza, grazie alla definizione che abbiamo dato, \\
se sia un insieme del codominio (corrispondente all'immagine della funzione) si può grazie alla controimmagine risalire a un altro insieme chiamato :
    \begin{center}
        $f^-1(B)$ con $B$ corrispondente all'immagine stessa
    \end{center}
Riuscendo cosi a determinare i valori del dominio mappati sul dominio.
    \item Si dice \textbf{fibra} invece: 
        \begin{center}
            $f^-1 (y) = \{x \in X | f(x) = y \}$
                \end{center}
Da questa definizione, possiamod definire la fibra semplicemente come : l'insieme dei punti nel dominio della funzione che vengono mappati in uno stesso punto del codominio.  \\
Facciamo un esempio per assimilarne al meglio il concetto di fibra :
    \begin{center}
        Sia una funzione costante $\rightarrow$ $f_c : \mathbb{R} \rightarrow \mathbb{R}$ \\[2EX]
        $f_c (\mathbb{R}) = \{C \}$, \\[1EX]
        $f_c ^-1 (D) = \{ \mathbb{R}$ se solo se $D = C$ \}, altrimenti sarà $\emptyset$
        
    \end{center}
In questo caso sia una fibra dove preso qualsiasi valore $x$ appartenente ad $\mathbb{R}$ il risultato della funzione sarà sempre $C$
\end{itemize}
\newpage
\subsubsection{Elenchiamo qualche proprietà delle immagini delle funzioni:} sia $f : x \rightarrow Y$, $A_1 \subseteq X$
\begin{enumerate}
    \item $f(A_1 \cup A_2) = f(A_1) \cup f(A_2)$
    \item $f(A_1 \cap A_2) \subseteq f(A_1) \cap f(A_2)$
    \item $A_1 \subseteq A_2 \implies f(A_1) \subseteq f(A_2)$
    \item $A \subseteq f^-1 (f(A))

\subsubsection{Esempio con Dirischle:}
sia $A_1 = [0,1]$, $A_2 = [1,2]$ \\
sia $A_1 \cap A_2 = \{1 \}$ \\[2EX]
Diremo quindi: 
    \begin{itemize}
        \item $dir(A_1 \cap A_2) = dir (\{1 \}) = \{1 \}$, con $1 \in \mathbb{Q}$
        \item $dir(A_1) = \{0, 1 \}, \ \ \ \ \ \ \ \ \sqrt{2}/2 \in A_1 \cap (\mathbb{R} \setminus \mathbb{Q})$
        \item $dir(A_2) = \{0, 1 \}, \ \ \ \ \ \ \ \ \sqrt{2} \in A_2 \cap (\mathbb{R} \setminus \mathbb{Q})$
    \end{itemize}

    
\end{enumerate} \\
\subsubsection{Elenchiamo adesso invece qualche proprietà della controimmagine :}
sia $f : x \rightarrow y$, \\
sia $B_1, B_2 \subseteq Y$, \\
\begin{enumerate}
    \item $f^-1 (B_1 \cup B_2) =  f^-1 (B_1) \cup f^-1 (B_2)$
    \item $f^-1 (B_1 \cap B_2) = f^-1 (B_1) \cap f^-1 (B_2)$
    \item $f^-1 (B_1 \setminus B_2) = f^-1 (B_1) \setminus f^-1 (B_2)$
    \item $f^-1 (B^c) = [f^-1(B)]^c$
    \item $f^-1 (B_1 \triangle B_2) = f^-1 (B_1) \triangle f^-1 (B_2)$
    \item $B_1 \subseteq B_2 \implies f^-1 (B_1) \subseteq f^-1 (B_2)$
    \item $f(f^-1 (B)) = B \cap f(x)$
    
\end{enumerate}

\newpage
\subsection{Iniettività e surriettività}
Definiamo = $f : x \rightarrow y$
    \begin{enumerate}
        \item $f$ è \textbf{suriettiva / surj} se e solo se $f(x) = Y$ \\
        dove : $(\forall y \in Y. \exists x \in X | f(x) = Y) \rightarrow$ $y$ è tutto il codominio
        \item $f$ è \textbf{iniettiva} se e solo se $\forall Y \in Y$ :
        \begin{center}
            $f^-1 (y)$ è $\emptyset$ oppure un singoletto, ovvero \textit{l'antimmagine} \\ dove ha al massimo 1 elemento
        \end{center}
        Se solo se $f(x) = f(x^I) \implies x = x^I$, oppure usando la forma della negazione : \\
        ( se $A \rightarrow B \equiv \perp B \rightarrow \perp A)$, \\[2EX]
        \begin{center}
             se $x \neq x^I \implies f(x) \neq f(x^1)
        \end{center}
       
        
    \end{enumerate}

\subsection{Biettività}
La funzione in questo caso sarà biettiva se è sia suriettiva che iniettiva:
    \begin{center}
        $\forall y \in Y. \exists ! x \in X | f(x) = y$
    \end{center}

\subsubsection{Proiezione canonica}
Le proiezioni canoniche sono funzioni che \textit{proiettano} elementi del prodotto cartesiano di due insiemi tramite $\pi$, facciamo un esempio:

        \begin{itemize}
            \item $\pi_1 proj_x = X \times Y \rightarrow X    $ \ \ \ \ \ \ \ \ \ \ $\pi_1 (x,y) = x$ \ \ \ \ \ \ \ \ \ \ Questa funzione restituisce il primo elemento della coppia.


            \item $\pi_2 proj_y = X \times Y \rightarrow Y     $ \ \ \ \ \ \ \ \ \ \ $\pi_1 (x,y) = y$ \ \ \ \ \ \ \ \ \ \ Questa funzione restituisce il secondo elemento della coppia.


            
        \end{itemize}
\newpage
\subsection{Composizione di funzioni}
Definiamo = $f: x \rightarrow y$ e $g : y \rightarrow z$ \\
Diciamo che $g \circ f : x \rightarrow z$, quindi diremo che : \ \ \ \ \ \ \ \ \ \ \ \ \ \  \ \ \ $(g \circ f) (x) = g(f(x))$



\begin{center}
    \begin{tikzpicture}[node distance=2cm]
    % Definire i nodi
    \node (X) {X};
    \node (Y) [right of=X] {Y};
    \node (Z) [right of=Y] {Z};

    % Disegnare le frecce
    \draw[->] (X) -- node[above] {$f$} (Y);
    \draw[->] (Y) -- node[above] {$g$} (Z);
    
    % Disegnare l'arco da X a Z
    \draw[->] (X) .. controls +(down:1cm) and +(down:1cm) .. (Z) node[midway, below] {$g \circ f$};
\end{tikzpicture}
\end{center}
Per fare un \textbf{esempio} scriveremo : 
Sia $quad : \mathbb{R} \rightarrow [0, +\infty)$
\begin{center}
    \begin{align*}
        &\sqrt{n} : [0, +\infty)] \rightarrow [0, +\infty] \\
        &\sqrt{n} \lor quad(x) = \sqrt{quad(x)} = \sqrt{x^2} = x
    \end{align*}
\end{center}
\subsubsection{Diamo qualche proprosizione utile:}

\begin{enumerate}
    \item $f \circ g \neq g \circ f$
    \item $h \circ (g \circ f) = (h \circ g) \circ f$
    \item su $f : x \rightarrow y$ sia :
        \begin{itemize}
            \item $f \circ id_x = f$
            \item $id_y \circ f = f$
        \end{itemize}
    
\end{enumerate}

\newpage
\subsection{Proprietà (inn, surr e comp)}
Sia $f : X \rightarrow Y$ e sia $g : Y \rightarrow Z$
    \begin{enumerate}
        \item se $f \land g$ sono iniettive \implies $g \circ f$ sono iniettive
        \item se $f \land g$ sono surrettive $\implies$ $g \circ f$ sono surrettive
        \item se $f \land g$ sono biettive $\implies$ $g \circ f$ sono biettive
        \item se $g \circ f$ è iniettiva $\implies$ $f$ è iniettiva
        \item se $g \circ f$ è suriettiva $\implies$ $g$ è suriettiva
    \end{enumerate}
\subsubsection{Proviamo a fare qualche esempio:}
    \begin{itemize}
        \item sia $f : \{0,1 \} \rightarrow \mathbb{R}$, con $0 \mapsto 1$ e con $1 \rightarrow \sqrt{2}$ \\
        $Dir : \mathbb{R} \mapsto \mathbb{R}$ (sappiamo che $Dir$ non è iniettiva) \\[2EX]
        Ma sappiamo che la composizione $(Dir \circ f)$ è iniettiva

\textbf{Perché \( Dir \circ f \) è iniettiva anche se \( Dir \) non lo è?}

1. \( Dir(f(0)) = Dir(1) \)
2. \( Dir(f(1)) = Dir(\sqrt{2}) \)

Se \( Dir(1) = Dir(\sqrt{2}) \), \\
ciò non implica che \( 1 = \sqrt{2} \). Pertanto, poiché \( 1 \) e \( \sqrt{2} \) sono due valori distinti in \( \mathbb{R} \), \( Dir(1) \) e \( Dir(\sqrt{2}) \) sono due uscite distinte della funzione \( Dir \).

\subsubsection{Conclusione}

La composizione \( Dir \circ f \) è iniettiva perché:

\begin{itemize}
    \item Ha solo due input possibili (\( 0 \) e \( 1 \)) che vengono mappati a due valori distinti \( Dir(1) \) e \( Dir(\sqrt{2}) \).
    \item La non iniettività di \( Dir \) non influisce sulla composizione \( Dir \circ f \) in questo caso, poiché gli argomenti \( f(0) \) e \( f(1) \) sono distinti e quindi mappati a valori distinti.
\end{itemize}
    \item sia $Dir : \mathbb{R} \rightarrow \mathbb{R}$ e sia $f_0 : \mathbb{R} \rightarrow \{0 \}$ \\
    $Dir$ non è suriettiva ma $f_0 \circ Dir$ lo è.
    \item sia $x = \mathbb{R} \setminus \{0\}$, \\
    poi $f : x \rightarrow x$, \\  $f(x) = x^2$
        \begin{center}
            Diremo che dato $g : x \rightarrow \mathbb{R}$ avremmo $g(z) = log(|z|)$
        \end{center}
    \end{itemize}
\newpage
\subsubsection{Risoluzione prorpietà d:}
se $g \circ f$ è iniettiva $\implies$ $f$ è iniettiva
    \begin{itemize}
        \item Per ipotesi sia $g \circ f$ iniettiva e consideriamo :
            \begin{center}
                $f(x) = f(x^I)$ allora \\
                Voglio provare che $x = x^I$
            \end{center}
        \item Prendo $g$ e calcolo su $f(x) \rightarrow g(f(x)) = g(f(x^I)$
            \begin{center}
                Per iniettività di $(g \circ f) \implies x = x^I$
            \end{center}
    \end{itemize}
\subsection{Proprietà della cancellabilità della funzione}
Sia una $f : x \mapsto y$, sono \textbf{equivalenti} le seguenti:
\begin{itemize}
    \item \begin{enumerate}
        \item $f$ è suriettiva
        \item $g,h : Y \mapsto Z$ \ \ \ \ \ $(\forall Z)$ \\
        $g \circ f = h \circ f \implies g = h$
    \end{enumerate}
    \item $f$ è iniettiva e sono equivalenti dire:
    \begin{enumerate}
        \item $f$ è inversa, quindi $\rightarrow$ $f : x \rightarrow y $
        \item $\forall Z$ sia $g,h : Z \rightarrow X$, questo se $\rightarrow$ $f \circ g = f \circ h \implies g = h$
    \end{enumerate}
\end{itemize}
    \begin{center}
        \begin{tikzpicture}
    % Nodi
    \node (X) at (0, 0) {$X$};
    \node (Y) at (3, 0) {$Y$};
    \node (Z) at (6, 0) {$Z$};

    % Collegamento orizzontale tra X e Y con etichetta f
    \draw[->] (X) -- (Y) node[midway, above] {$f$};

    % Collegamenti ad arco tra Y e Z con etichette g e h
    \draw[->, bend left=50] (Y) to node[above] {$g$} (Z);
    \draw[->, bend right=50] (Y) to node[below] {$h$} (Z);

    % Collegamenti ad arco tra X e Z con etichette g composto f e h composto g
    \draw[->, bend left=80] (X) to node[above] {$g \circ f$} (Z);
    \draw[->, bend right=80] (X) to node[below] {$h \circ g$} (Z);

\end{tikzpicture}
    \end{center}
In questo caso di dice che $f$ è cancellabile a destra

\newpage
\subsection{Funzione inversa di biettiva}
Se una funzione \( f: A \rightarrow B \) è biettiva, significa che è sia iniettiva (ogni elemento di \( A \) mappa a un elemento unico in \( B \)) sia suriettiva (ogni elemento di \( B \) ha almeno un elemento di \( A \) che lo mappa). In altre parole, per ogni \( b \in B \), esiste un unico \( a \in A \) tale che \( f(a) = b \).

La \textbf{funzione inversa} di una funzione biettiva \( f \), indicata con \( f^{-1} \), è una funzione che "inverte" l'effetto di \( f \). In altre parole, l'inversa è una funzione \( f^{-1}: B \rightarrow A \) tale che:

\[
f^{-1}(f(a)) = a, \quad \forall a \in A
\]
e
\[
f(f^{-1}(b)) = b, \quad \forall b \in B.
\]

Questo significa che, applicando la funzione \( f \) e poi la sua inversa \( f^{-1} \) (o viceversa), si ottiene l'elemento di partenza. \\
Ecco alcuni esempi di funzioni biettive e delle loro inverse:

\begin{enumerate}
    \item \textbf{Funzione lineare semplice}: La funzione \( f(x) = 2x + 3 \) è biettiva sull'insieme dei numeri reali \( \mathbb{R} \).
    \begin{itemize}
        \item Inversa: \( f^{-1}(y) = \frac{y - 3}{2} \).
    \end{itemize}
    
    \item \textbf{Funzione esponenziale}: La funzione \( f(x) = e^x \) è biettiva da \( \mathbb{R} \) a \( (0, +\infty) \).
    \begin{itemize}
        \item Inversa: \( f^{-1}(y) = \ln(y) \), con \( y > 0 \).
    \end{itemize}
    
    \item \textbf{Funzione radice quadrata}: La funzione \( f(x) = x^2 \) non è biettiva su \( \mathbb{R} \), ma è biettiva se considerata solo su \( x \geq 0 \).
    \begin{itemize}
        \item Inversa: \( f^{-1}(y) = \sqrt{y} \), con \( y \geq 0 \).
    \end{itemize}
\end{enumerate}

Vediamo come si determina l'inversa della funzione lineare \( f(x) = 2x + 3 \) passo dopo passo.

\subsubsection{Passi per trovare l'inversa di una funzione lineare}

\begin{enumerate}
    \item \textbf{Partire dalla funzione}: 
    \[
    y = f(x) = 2x + 3.
    \]

    \item \textbf{Scambiare \( x \) e \( y \)}: 
    L'inversa della funzione viene trovata scambiando \( x \) e \( y \). Quindi, riscriviamo l'equazione come:
    \[
    x = 2y + 3.
    \]

    \item \textbf{Isolare \( y \)}:
    Ora dobbiamo risolvere l'equazione per \( y \):
    \begin{itemize}
        \item Sottrai 3 da entrambi i lati:
        \[
        x - 3 = 2y.
        \]
        \item Dividi entrambi i lati per 2:
        \[
        y = \frac{x - 3}{2}.
        \end{itemize}
    \end{enumerate}

    \item \textbf{Scrivere l'inversa}:
    Ora che abbiamo \( y \) in funzione di \( x \), possiamo scrivere l'inversa come:
    \[
    f^{-1}(x) = \frac{x - 3}{2}.
    \]

\subsubsection{Elenchiamo adesso qualche prorpietà per aiutarci nelle dimostrazioni:}
\begin{enumerate}
    \item se $f : x \rightarrow y$, $g : y \rightarrow x$, $h : y \rightarrow x$ tale che:
        \begin{center}
            $(g \circ f) = id_x$ e $(f \circ h) = id_y$
        \end{center}
    Allora diremo che $g = h$
    \item \textbf{Prorpietà di equivalenza}, si definisce come :
        \begin{itemize}
            \item $f : x \rightarrow y$, è biettiva
            \item $f$ ha l'inversa $g : y \rightarrow x$
            
        \end{itemize}
    Diremo che $f$ è biettiva $\implies f^-1$ biettiva.
\end{enumerate}

\subsubsection{Proviamo a fare una piccola dimostrazione per avere una prospettiva diversa di questi concetti:}
sia, se $f$ è invertibile (ammette l'inversa) allora è biettiva.
\begin{itemize}
    \item $\implies)$ sia $f$ invertibile per ipotesi
    \begin{center}
        $\exists g \; | \; g \circ f = id_x$ e $f \circ g = id_y$ \\

        \begin{array}{rl}
            \text{con } a + b = f \text{ biettiva} & \left\{
            \begin{array}{l}
                a = id_x \quad \text{è iniettiva} \implies f \text{ è iniettiva} \\
                b = id_y \quad \text{è suriettiva} \implies f \text{ è suriettiva}
            \end{array}
            \right.
        \end{array}
    \end{center}
    \item $\impliedby )$ sia $f$ biettiva per ipotesi
    \begin{center}
        $\forall y \in Y, \exists x \in X | f(x) = y$ e $x$ è \textit{unico} (per surriettività)
    \end{center}
    - Poniamo quindi $g(y) = x$
        \begin{center}
            $f(g(y)) = f(x) = y \rightarrow \forall y$ \\
            $\implies f \circ g = id_Y \rightarrow |\forall x \in X$
        \end{center}
    - Poniamo adesso: \begin{align*} &g(f(x)) = x \\
                                    &g \circ f = id_X
    \end{align*}
    - Avremmo quindi tramite $f \circ g = id_Y$ e $g \circ f = id_X \rightarrow g = f^-1$ 
\end{itemize}

\newpage
3,14 \\[25ex]
\begin{center} 
    {\scalebox{50}{$\pi$}}  % Scala il simbolo pi greco 10 volte
\end{center}
\newpage

\section{Equivalenza e partizioni}
\subsection{Equivalenza}
Definiamo = $R \subseteq X$, prendendo $X$ come un insieme;
    \begin{enumerate}
        \item $R$ è \textbf{riflessiva} se per ogni elemento di x, ogni elemento è in relazione con se stesso
        \item $R$ è \textbf{simmetrica} se $\forall x, y \in X$ e se $xRy$, allora diremmo che: $yRx$
        \item $R$ è \textbf{transitiva} se $\forall x, y, z \in X$ e:
            \begin{center}
                - se sia $xRy$ e sia $yRz$, \\
                - Allora diremmo che $\rightarrow xRz$
            \end{center}
        \item $R$ è quindi di \textbf{equivalenza} se e solo se \textit{valgono \textbf{tutte} le proprietà precendenti}
    \end{enumerate}
\subsubsection{Proviamo a fare qualche esempio :}
\begin{itemize}
    \item \textbf{Es1)} $\rightarrow R = \{(x,y) \in \mathbb{N}^2 | x + y$ è pari\}
        \begin{enumerate}
            \item Se $R$ è riflessiva, $xRx \iff x + x$ è pari
            \item $R$ è simmetrica se $x + y$ è pari, quindi $\implies y + x $ è pari
            \item $R$ è transitiva se $xRy \and yRz \implies xRz$ \\
                    - Questo solo se: \begin{align*}
                        &x+y = \texttt{pari} \\
                        &y+z = \texttt{pari} \\
                        &x+z = \texttt{pari}
                    \end{align*}
                    - Allora diremmo che $R$ è di \textit{EQUIVALENZA}
        \end{enumerate}
    \item \textbf{Es2)} $\rightarrow X \sim Y$, e $Y, X$ sono arbitrari, \\
    \textit{(sono detti arbitrari quando non hanno particolari restrizioni o caratteristiche specifiche che li definiscono. In altre parole, si usano due insiemi arbitrari quando si vuole parlare di proprietà o relazioni che valgono per qualsiasi insieme, indipendentemente dalla loro composizione, dimensione o elementi specifici.)} \\
        \begin{enumerate}
            \item $X \sim Y \iff \exists f : x \rightarrow y$ (biettiva fra due insiemi)
            \item $X \sim Y$ è di equivalenza se :
                \begin{itemize}
                    \item 1) riflessiva = $X \sim X$, si se prendo $id_X$
                    \item 2) simmetrica = $X \sim Y \land Y \sim X$, poichè $f$ è biettiva posso considerare $f^-1$
                    \item 3) transitiva = tramite la proprietà della composizione e biezione :
                        \begin{center}
                            $X \sim Y$ \\
                            $Y \sim Z$
                        \end{center}
                \end{itemize} 
\newpage
            \item \textbf{sulla proiezione canonica invece:}
                \begin{itemize}
                    \item \textbf{nota 1}
                        \begin{center}
                            $\pi_\sim$ è  suriettiva
                        \end{center}
                    \item \textbf{nota 2} $\sim$, sia $\pi_\sim$ proiezione canonica
                        \begin{center}
                            \textbf{costruisco la relazione indotta:}\\
                                \begin{itemize}
                                    \item $X_1 R_\pi_\sim X_2 = \pi_\sim (X_1) = \pi_\sim (X_2)$
                                    \item $[X_1]_\sim = [X_2]_\sim \iff X_1 \sim X_2 $
                                    \item $R_\pi_\sim = \sim$
                                \end{itemize}
                        \end{center}
                \end{itemize}
        \end{enumerate}
\end{itemize}
\subsubsection{Identità di X su X:}
Sia :
    \begin{center}
        $[X]id_X = \{X \}$ \\
        $X/id_X = \{\{X\} | x \in X\}$
        
    \end{center}
\subsubsection{Proprietà:} Sia $X$ un insieme, $\sim$ su $X$, $(x, y \in X)$ :
    \begin{enumerate}
        \item $x \in [X]_\sim$
        \item $[x] = [y]_\sim$ se e solo se $x \sim y$
        \item $[x]_\sim \neq [y]_\sim \iff [x]_\sim \cap [y]_\sim = \emptyset$
        \item $\pi_\sim : X \rightarrow X/\sim$  \textbf \rightarrow {suriettiva}$    \end{enumerate}
Prima di passare alle \textit{partizioni}, enunciamo le ultime proposizioni sull'equivalenza: \\
\textbf{Le seguenti sono equivalenti :}
    \begin{enumerate}
        \item $R_1 = R2$
        \item $[X]_R_1 = [X]_R_2 \rightarrow \forall x \in X$
        \item $X/R1 = X/R2$
    \end{enumerate}
\newpage


\subsection{Partizioni di un insieme}
Sia $X \neq \emptyset$
Per definizione una \textbf{partizione} è una famiglia \( \mathcal{F} \) di sottoinsiemi di $X$
   \begin{center}
    \(
    \mathcal{F} = \{A_i \mid i \in I \} \rightarrow 
    \begin{aligned}
        &A_i \subseteq X \\
        &i \in I
    \end{aligned}
    \)
\end{center}
Tale che: 
    \begin{enumerate}
        \item $A_i \neq \emptyset \rightarrow \forall i \in I$
        \item $\bigcup\limits_{i \in I} A_i = X$
        \item se $A_i$ $\neq A_j \implies A_i \cap A_j = \emptyset$ \ \ \ \ \ \ \ \ \ \ \ \ \ \ \ \ \ \ \ \ \ \ \ \ \ \ \ \ \ \ \ \ \ $\rightarrow A_ij \in I$
    \end{enumerate}


\begin{center}
\begin{tikzpicture}
    % Disegna il contorno dell'insieme A
    \draw[thick] (0, 0) circle (3);
    \node at (3.5, 2.5) {\( A \)};
    
    % Linee di separazione per suddividere il cerchio
    \draw[thick] (0, -3) -- (0, 3); % Linea verticale
    \draw[thick] (-3, 0) -- (3, 0); % Linea orizzontale
    
    % Etichette delle partizioni
    \node at (-1.5, 1.5) {\( A_1 \)};
    \node at (1.5, 1.5) {\( A_2 \)};
    \node at (-1.5, -1.5) {\( A_3 \)};
    \node at (1.5, -1.5) {\( A_4 \)};
\end{tikzpicture}
\end{center}
Facciamo qualche esempio: \\
\textbf{1) Partizione totale:} $\rightarrow$ $\mathcal{F} = \{X\}$ \\
\textbf{2)} $u$ identica $\mathcal{F}_id = \{\{X\} | x \in X\}$ \\
\textbf{3)} $\mathbb{N} = \{\mathbb{N}_d, \mathbb{N}_p\}, \ \ \ \ \ \ \{\{0\}, \mathbb{N}^+\}$ \\[1EX]

Possiamo quindi dire che La \textbf{partizione di un insieme} è una suddivisione dell'insieme in sottoinsiemi disgiunti che coprono completamente l'insieme originale.\\ 
In altre parole, una partizione divide un insieme in gruppi in modo che ogni elemento dell'insieme appartenga esattamente a un solo gruppo.
\newpage

\subsubsection{Esempio:}
Considera l'insieme $A = \{1, 2, 3, 4, 5\}$. Una possibile partizione di $A$ potrebbe essere:
\begin{itemize}
    \item $A_1 = \{1, 2\}$
    \item $A_2 = \{3\}$
    \item $A_3 = \{4, 5\}$
\end{itemize}

In questo caso:
\begin{itemize}
    \item $A_1 \cap A_2 = \emptyset$
    \item $A_1 \cap A_3 = \emptyset$
    \item $A_2 \cap A_3 = \emptyset$
    \item $A = A_1 \cup A_2 \cup A_3$.
\end{itemize}

\subsubsection{Attenzione:}
$\mathcal{F}$ partizione $\implies \exists !$ equivalenza $R_\mathcal{F}$ \\
Abbiamo una \textbf{biezione} tra l'insieme delle relazioni di equivalenza su $X$ e le sue partizioni.\\
\subsubsection{Proprietà:}
sia $X \neq \emptyset$:
    \begin{enumerate}
        \item se $\sim$ di equivalenza su $X$, allora l'insieme $X/\sim$ è una partizione di $X$
        \item sia $\sim_\mathcal{F}$ definita su $x,y \in X$
            \begin{center}
                $X \sim Y \iff \exists i \in I$ tale che $(x \in A_i, \& y \in A)$ \textit{è di equivalenza}
            \end{center}
    \end{enumerate}


\end{document} 
